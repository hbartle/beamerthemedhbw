%%%%%%%%%%%%%%%%%%%%%%%%%%%%%%%%%%%%%%%%%%%%%%%%%%%%%%%%%%%%%%%%%%%%%%%%%%%%%%%%%%%%%%%%%%
%%
%% Description:		This is an example presentation using the beamerthemedhbw
%%
%%					The beamerthemedhbw is based on jacksbeamertheme
%%					(https://github.com/JacknJo/jacksbeamertheme)
%%
%% Author:			Hannes Bartle																				
%% 					DHBW Ravensburg Campus Friedrichshafen		
%%					September 2016	
%% 
%% The beamerthemedhbw is free software: you can redistribute it and/or modify
%% it under the terms of the GNU General Public License as published by
%% the Free Software Foundation, either version 3 of the License, or
%% (at your option) any later version.
%% 
%% The beamerthemedhbw is distributed in the hope that it will be useful,
%% but WITHOUT ANY WARRANTY; without even the implied warranty of
%% MERCHANTABILITY or FITNESS FOR A PARTICULAR PURPOSE.  See the
%% GNU General Public License for more details.
%% 
%% You should have received a copy of the GNU General Public License
%% along with the beamerthemedhbw.  If not, see <http://www.gnu.org/licenses/>.
%% 
%% 
%%%%%%%%%%%%%%%%%%%%%%%%%%%%%%%%%%%%%%%%%%%%%%%%%%%%%%%%%%%%%%%%%%%%%%%%%%%%%%%%%%%%%%%%%%


\documentclass[	12pt, 				
				t,					
				aspectratio=169,
				%handout
				]{beamer}


\usepackage[utf8]{inputenc}
\usepackage[T1]{fontenc}


\title{This will be a cool Title }
\date{Friedrichshafen\\\today}
\author{Hannes Bartle \texttt{this-is-a-mail@adress-hippie.yeah}}
\institute{DHBW Ravensburg \\ Campus Friedrichshafen}

\usetheme{dhbw}
\renewcommand{\usecompanylogo}{1}

\newcommand{\printcompanylogo}{\includegraphics[width=4cm]{example/airbus_logo.png}}

\newcommand{\middleText}{Hannes Bartle}
\newcommand{\outlineSection}{Outline}

\setlength{\framesubtitleoffset}{0pt}
\setlength{\frametitleoffset}{0.05cm}
\setlength{\frametextoffset}{0.1cm}
\setlength{\framefootlineoffset}{0cm}


\begin{document}
	
	\begin{frame}[noframenumbering]
		\titlepage
	\end{frame}


	\begin{frame}{Outline}
		\tableofcontents
	\end{frame}


	\outlineFrame{Introduction}
	
	
	\begin{frame}{Background}{\& personal inspiration}
		Hello all of you! 
		
		I'm Jack, 20 years old and student in electrical engineering in Friedrichshafen in Germany.
		This is my first try to design a own \LaTeX{} template.
		
		It's advantages are:
		\begin{itemize}
			\item clean structure
			\item nice \& consistent design
			\item easy to use with a small basic \LaTeX{} knowledge
			\item useful additional commands
			\item export file is \texttt{.pdf} which can be opened nearly anywhere!
		\end{itemize}
	\end{frame}
	
	
	\outlineFrame{Utilities}
	
	
	
	\begin{frame}{Itemize}{}
		One of the most common ways to create slides is to use lots of itemizes.
		
		They are used for:
		\begin{itemize}
			\item better readability
			\item as short descriptions
			\begin{itemize}
				\item for easy points
				\item or difficult
				\begin{itemize}
					\item topics
					\item statements
				\end{itemize}
			\end{itemize}
			\item <2-> they also can be made visible item by item
			\item <3-> like this!
		\end{itemize}
	\end{frame}
	
	
	
	\begin{frame}{Math environment}{presenting formula is very easy in \LaTeX}
		\vfill
		\begin{equation*}
			f(x)=\sum_{i=0}^\infty \frac{f^{(i)}(x_0)}{i!}(x-x_0)^i
		\end{equation*}
		\vfill
		\begin{equation*}
			\displaystyle\frac{\pi^2}{6}=\lim_{n \to \infty}\sum_{k=1}^n \frac{1}{k^2}
		\end{equation*}
		\vfill
	\end{frame}
	
	
	\begin{frame}{Blocks}{example, alert and standard}
	    \begin{exampleblock}{Exampleblock}
			Inside this box you can place an example or a task for your audience:
			
			What is the square root of 144? $\sqrt{144} = $
		\end{exampleblock}
		\begin{alertblock}{Alertblock}			
			There is no real solution to a negative square root!
		\end{alertblock}
		\begin{block}{Block}
			Use this for any other reason you can imagine!
		\end{block}
	\end{frame}
	
	
	\begin{frame}{Comparison}{of advantages and disadvantages}
		This is a example comparison:
		\begin{center}
		\begin{tabularx}{0.8\textwidth}{X|X}
			\textbf{Advantages}
			 \begin{itemize}[<2->]
			 	\item lifetime
			 	\item easy handling
			 	\item costs
			 \end{itemize} &
			\textbf{Disadvantages}
			 \begin{itemize}[<3->]
			 	\item weight
			 	\item resolution
			 	\item integration time
			 \end{itemize}
		\end{tabularx}
		\end{center}
	\end{frame}
	
	\begin{frame}{Comparison}{of two items}
		This is a example comparison:
		\begin{center}
		\begin{tabularx}{\textwidth}{XcX}
			\textbf{Advantages} & & \textbf{Disadvantages}\\
			lifetime & \drawopp & weight \\
			easy handling & \drawopp & resolution \\
			costs & \drawopp & integration time (also works for long texts)\\
			another & \drawopp & ... \\
		\end{tabularx}
		\end{center}
	\end{frame} 
	
	
	\begin{frame}{Todo}{list of all tasks}
		Many projects have deadlines and tasks you have to fulfill by a certain date.
		Presenting them the following way is very easy and leaves a good and professional impression.
		\vfill
		\begin{tabularx}{\textwidth}{Xll}
			\textbf{task} & \textbf{deadline} & \textbf{state}\\
			basic tutorials in creating a beamer template & 01.02.03 & \myCheck \\
			starting a own development & 01.03.16 & \myCheck \\
			creating a presentation with all the theme's capabilities & 03.04.16 & \myCheck \\
			finishing the beamer template & 01.06.16 & \myUnCheck\\
		\end{tabularx}
		\vfill
	\end{frame}
	
		
	\section*{The End}
	\begin{frame}
		\vspace{2.2cm}
		\begin{center}
			\huge Thank you for watching this presentation.\\
			Have fun creating your own!
		\end{center}
		\vfill
	\end{frame}
	

\end{document}